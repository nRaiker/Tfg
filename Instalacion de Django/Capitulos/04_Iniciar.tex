\section{Templates}

Si queremos que nuestro servidor web sea accesile para otras máquinas, primero debemos
modificar el archivo settngs.py. Buscamos las líneas donde aparecen DEBUG y ALLOWED\_HOST y 
las dejamos de la siguiente manera


\lstset{
  language=Python,
}
\begin{lstlisting}
# SECURUTY WARNING: don't run with debug turned on in production!
DEBUG = False

ALLOWED_HOSTS = ['*']


\end{lstlisting}

Con DEBUG=False hacemos que si ocurre un error no aparezca la información de depuraciom en el 
navegador, mostrando solamente un mensaje de error

En ALLOWED\_HOST podemos añadir las IPs que nos interesessi queremos restringir el acceso a un
conjunto de máquinas, o '*' si queremos que pueda acceder cualquier equipo.

Para iniciar el servidor, ahora nos movemos al diretorio en el que se encuentra el archivo manage.py y ejecutamos el siguiente comando

\lstset{
  language=Bash,
}
\begin{lstlisting}
python3.5 manage.py runserver 0.0.0.0:<puerto>
\end{lstlisting}

Y si no ha ocurrido ningun error ya tenemos el servior web abierto en el puerto indicado.

