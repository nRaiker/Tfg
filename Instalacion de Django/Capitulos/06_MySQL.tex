\section{MySQL en Django}

Para utilizar MySQL primero debemos instalar los siquientes paquetes:

\lstset{
  language=Bash,
}
\begin{lstlisting}
sudo apt-get update
sudo apt-get install python-pip python-dev mysql-server libmysqlclient-dev
sudo pip install MySQL-python
\end{lstlisting}


En el transcurso de la instalación se nos preguntara por la contraseña que queremos darle al usuario root de MySQL

Ahora debemos acceder a la base de datos para hacer unos ajustes, para ello usamos el comando


\begin{lstlisting}
mysql -u root -p
\end{lstlisting}

Ahora debemos crear una base de datos para nuestro proyecto (es aconsejable que cada proyecto tenga su propia base de datos) con el comando


\lstset{
  language=sql,
}
\begin{lstlisting}
CREATE DATABASE lmd CHARACTER SET UTF8;
\end{lstlisting}

Y creamos un usuario para que interactue con la base de datos

\begin{lstlisting}
CREATE USER nraik@localhost IDENTIFIED BY '111111';
\end{lstlisting}


Y le damos todos los privilegios de la base de datos del proyecto

\begin{lstlisting}
GRANT ALL PRIVILEGES ON lmd.* TO nraik@localhost;
\end{lstlisting}
Y guardamos los cambios realizados

\begin{lstlisting}
FLUSH PRIVILEGES;
\end{lstlisting}


Ahora tenemos que indicarle al proyecto de Django que vamos a usar MySQL. Para ello buscamos en settings.py la seccion DATABASES y la modificamos de la siguiente manera:

\lstset{
  language=Python,
}
\begin{lstlisting}
DATABASES = {
    'default': {
        'ENGINE': 'django.db.backends.mysql',
        'NAME': 'lmd',
        'USER': 'nraik',
        'PASSWORD': '111111',
        'HOST': 'localhost',
        'PORT': '',
       }
}
\end{lstlisting}

Por podemos migrar las estructuras de a base de datos antigua de la siguiente forma

\lstset{
  language=Bash,
}
\begin{lstlisting}
python3.5 manage.py makemigrations
python manage.py migrate

\end{lstlisting}

Y despues creamos el usuario administrador de la siguiente forma

\begin{lstlisting}
python manage.py createsuperuser
\end{lstlisting}

Para comprobar que se ha realizado el cambio podemos acceder a la página http://localhost:<puerto>/admin y comprobar que el usuario creado con la orden anterior
y su contraseña funcionan
