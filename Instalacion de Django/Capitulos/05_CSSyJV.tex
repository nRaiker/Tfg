\section{Añadir css y javacript}

Para utilizararchivos .cc y javascript en un proyecto Django, debemos crear una carpeta
llamada static en el mismo lugar donde se encuentra la carpeta templates. Dentro de esta
carpeta creamos los directorios que queramos para organizar los distintos tipos de archivos.

Ahora debemos modificar el archivo settings.py, la liena que contiene STATIC\_URL y añadimos
esto al final

\lstset{
  language=Python
}
\begin{lstlisting}

# Static files (CSS, JavaScript, Images)
# https://docs.djangoproject.com/en/1.10/howto/static-files/
STATIC_URL = '/static/'

STATICFILES_DIRS = [
    os.path.join(BASE_DIR, "static"),
    '/var/www/static/',
]

\end{lstlisting}

Por ultimo, un ejemplo de como añadir un script de javascript en un template:

\lstset{
  language=HTML
}
\begin{lstlisting}

<script type="text/javascript" src="></script>

\end{lstlisting}