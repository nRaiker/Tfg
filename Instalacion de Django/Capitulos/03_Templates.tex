\section{Templates}

Un template de Django es una cadena de texto que se utiliza para separar la representación
de la página de los datos. Contienen también algunas reglas lógicas para regular
que partes de la página se tienen que mostar y se encarga de generar el código html. La
forma más comoda y ordenada de hacerlo es crear para cada página un fichero html dentro
de la carpeta templates. Para poder utilizar este método tenemos que modificar el archivo
settings.py (podemos ver donde se encuentra en la figura 1). Buscamos la parte
correspondiente a los templates, que es de la siguiente forma:


\lstset{
  language=Python,
}
\begin{lstlisting}
TEMPLATES = [
    {
        'BACKEND': 'django.template.backends.django.DjangoTemplates',
        'DIRS': [],
        'APP_DIRS': True,
        'OPTIONS': {
            'context_processors': [
                'django.template.context_processors.debug',
                'django.template.context_processors.request',
                'django.contrib.auth.context_processors.auth',
                'django.contrib.messages.context_processors.messages',
            ],
        },
    },
]
\end{lstlisting}

Modificamos la linea DIRS para que quede de la siguiente forma

\begin{lstlisting}
'DIRS': [os.path.join(BASE_DIR, 'templates')],
\end{lstlisting}

De esta forma ya podemos utilizar la carpeta templates, creada en el apartado 1 para
almacenar todos nuestros templates.

\hspace{1cm}

Para más información sobre los templates en django:

\hspace{1cm}

http://djangobook.com/django-templates/

http://djangobook.com/basic-template-tags-filters/

http://djangobook.com/templates-in-views/

http://djangobook.com/advanced-templates/

