\chapter{Introducción}


Tras la publicación del Real Decreto 3/2010, del 8 de enero, se regulo el Esquema Nacional de Seguridad en el ámbito 
de la Administración Electrónica. Este documento se creo con la finalidad de crear las condiciones necesarias en el
uso de los medios electrónicos a traves de distintas medidas de seguridad, permitiendo así el correcto funcionamiento de
las Administraciones públicas.


Dentro del esquema, en el Anexo 11: Medidas de seguridad, encontramos un apartado dedicado a la limpieza de documentos
[5.7.6], en el cual se índica que se debe eliminar cualquier información adicional contenida en campos ocultos, 
meta-datos y comentarios o revisiones anteiores, salvo cuando dicha información sea pertinente para el receptor del 
ducumento.

Como se nos indica tambíen en el Esquema, el incumplimiento de esta medida puede perjudicar:

\begin{enumerate}
	\item Al mantenimiento de la confidencialidad de información que no debería haberse revelado al receptor del
	documento.
	\item Al mantenimiento de la confidencialidad de las fuentes u origenes de la información, que no debe 
	conocer el receptor del documento.
	\item A la buena imagen de la organización que difunde el documento por cuanto demuestra un descuido en su buen
	hacer.
\end{enumerate}



\section{Metadatos: Lo que ocultan los documentos}

Cuando hablamos sobre metadatos, nos referimos a aquellos datos de un documento que se refieren a las propiedades del 
mismo. Cada tipo de documento suele contener distintos tipos de metadatos. Un documento ofimático por ejemplo puede 
contener información sobre la fecha de creación y de última modificación, autor, historial de versiones, etc, mientras 
que una fotografía puede contener otros datos como son la orientación o la geolocalización.

La divulgación de esta información puede no ser peligrosa para nuestro sistema, aunque esto no quita que se este
ignorando la confidencialidad de cierta información al compartirla adjunta a los metadatos de los documentos. Ademas, 
esta información puede comprometer otros sectores de seguridad, pudiendose revelar vulnerabilidades del sistema. Un 
ejemplo sería revelar un sofware y su versión del mismo, con lo que la busqueda de vulnerabilidades por un atacante 
podría ser menos costosa.


\section{Responsabilidad sobre los metadatos}

Como se indica en el artículo 12 del Esquema Nacional, la seguridad deberá comprometer a todos los miembros de la 
organización. Es por ello que son los autores de los documentos que se van a compartir por la red y los que los 
comparten responsables de cumplir con la limpieza de los documentos. En el articulo 5 también se dice que la seguridad 
se entendera como un proceso integral constituido todos los elementos técnicos, humanos, materiales y organizativos, 
relacionados con el sistema. Con estos dos articulos también podemos entender que la organización en si también es en 
parte responsable de que se cumpla el Esquema y que es favorable que ponga a disposición recursos para facilitar su 
cumplimiento, como puede ser herramientas para la localización y limpieza de
metadatos.

Por último, cabe destacar que aunque parezca que la limpieza de documentos puede parecer una tarea menos importante que 
otras de las que aparecen en el Esquema, tal como se indica en el documento la debilidad de un sistema la determina su 
punto más frágil, por lo que no se debe descuidar ningún aspecto en cuanto a seguridad se refiere.


\section{Objetivos}

El objetivo de este proyecto es proporcionar una plataforma a la universidad de Granada para que su
personal pueda limpiar de meta-datos los documentos que van a compartir o subir a la red, conforme a lo establecido
en el Esquema Nacional de Seguridad. Este proyecto me fué propuesto por Antonio Muñoz Ropa, jefe de servicio del area
de seguridad del CSIR, mientras realizaba unas practicas de empresa en el area.

\subsection{Objetivos principales}



\section{Referencias}

Esquema nacional de seguirdad

1.1:

http://blog.isecauditors.com/2014/09/los-metadatos-definicion-riesgo-eliminacion.html